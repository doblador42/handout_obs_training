\documentclass[12pt,a4paper]{article}
\usepackage{fullpage}
% Language and font settings
\usepackage{polyglossia}
\setdefaultlanguage{greek}      % Main language
\setotherlanguage{english}      % Secondary language
\usepackage{fontspec} % Required for font selection in lualatex
\usepackage{csquotes}
\usepackage{libertine}
% Formatting and lists
\usepackage{enumitem} % Customizable lists
\setlist{noitemsep}   % Remove extra space between list items
% Boxes and highlights
\usepackage{tcolorbox}
\usepackage{hyperref}
\usepackage{parskip} % No paragraph indentation
\hypersetup{
  colorlinks=true,
  linkcolor=blue,
  filecolor=magenta,
  urlcolor=cyan,
  citecolor=blue,
  pdfpagemode=FullScreen,
}
\tcbuselibrary{listings, breakable}

\newtcolorbox{info}{colback=blue!5!white, colframe=blue!75!black}
\newtcolorbox{questionbox}{colback=green!5!white, colframe=green!75!black, breakable}
\usepackage{amsmath, amssymb} % Math support
\title{\textbf{Handout για εξάσκηση το OBS}}
\author{Ερμής Δούλος (\href{mailto:dit17046@uop.gr}{dit17046@uop.gr})\thanks{Επιβλέπων καθηγητής: Ιωάννης Μοσχολιός (\href{mailto:idm@uop.gr}{idm@uop.gr})}}
\date{\today}

\begin{document}
\maketitle
\begin{center}
  \includegraphics[width=0.5\textwidth]{Figures/uop.png}
\end{center}
\tableofcontents
\noindent\makebox[\linewidth]{\rule{\textwidth}{1.5pt}}

\section*{Εισαγωγή}

Στο σημερινό εργαστήριο, θα εμβαθύνουμε στο \textbf{OBS Studio}, ένα ισχυρό εργαλείο
για ζωντανή αναμετάδοση και καταγραφή περιεχομένου. Μέσα από πρακτική εξάσκηση,
κάθε συμμετέχων θα μάθει πώς να δημιουργεί μια σκηνή με κάμερα και μικρόφωνο, να
καταγράφει υψηλής ποιότητας βίντεο, και να πραγματοποιεί ζωντανές μεταδόσεις στο
YouTube.

\subsection*{Τι θα κάνουμε αναλυτικά:}

\begin{itemize}
  \item \textbf{Γνωριμία με το OBS Studio:} Θα εξετάσουμε το περιβάλλον χρήστη του
        OBS, εστιάζοντας στις ενότητες \textit{Scenes}, \textit{Sources}, \textit{Audio
          Mixer}, και \textit{Settings}. Θα εξηγήσουμε πώς αυτές συνεργάζονται για τη
        δημιουργία και μετάδοση περιεχομένου.

  \item \textbf{Δημιουργία σκηνών:} Ο καθένας θα δημιουργήσει μια σκηνή, όπου θα
        προσθέσει μια κάμερα και ένα μικρόφωνο. Θα ρυθμίσουμε την ανάλυση βίντεο και τις
        παραμέτρους ήχου ώστε να εξασφαλίσουμε καθαρή και επαγγελματική ποιότητα.

  \item \textbf{Καταγραφή βίντεο:} Θα ασχοληθούμε με τη διαμόρφωση των ρυθμίσεων
        καταγραφής, όπως το bit rate και το format του αρχείου. Στόχος είναι να κατανοήσουμε
        πώς να παράγουμε βίντεο κατάλληλα για αποθήκευση ή επεξεργασία.

  \item \textbf{Ζωντανή μετάδοση στο YouTube:} Στη συνέχεια, θα συνδέσουμε το OBS
        Studio με το YouTube μέσω RTMPS κλειδιού\footnote{Real-Time Messaging Protocol Secure}. Θα εξηγήσουμε τις βασικές παραμέτρους
        μετάδοσης και ο καθένας θα έχει την ευκαιρία να πραγματοποιήσει τη δική του
        ζωντανή μετάδοση\footnote{Ένας συμμετέχων τη φορά, για να βλέπουμε όλοι τη δημιουργία του}.
\end{itemize}

\section{Αναλυτικές Οδηγίες}
Παρακάτω θα δοθούν αναλυτικές οδηγίες για την εκτέλεση των προαναφερθέντων εργασιών.
\subsection{Εγκατάσταση του OBS Studio}
Αρχικά, θα πρέπει να εγκαταστήσετε το OBS Studio στον υπολογιστή σας.
Μπορείτε να το κατεβάσετε από την επίσημη ιστοσελίδα του OBS (\url{https://obsproject.com/}).
\subsection{Γνωριμία με το OBS Studio}
Μόλις ανοίξετε το OBS Studio, θα δείτε το κεντρικό παράθυρο της εφαρμογής. Αυτό το παράθυρο
αποτελείται από διάφορες ενότητες, όπως οι \textit{Scenes}, \textit{Sources}, \textit{Audio Mixer
  και \textit{Settings}}. Κάθε μια από αυτές τις ενότητες έχει το δικό της ρόλο στη δημιουργία
και μετάδοση περιεχομένου. Στη συνέχεια, θα εξηγήσουμε τις βασικές λειτουργίες του κάθε τμήματος.
\subsubsection{Scenes}
Οι σκηνές είναι οι διάφορες προβολές που μπορείτε να δημιουργήσετε στο OBS Studio. Κάθε σκηνή
περιέχει διαφορετικές πηγές, όπως κάμερες, οθόνες, εικόνες, και ήχο. Μπορείτε να δημιουργήσετε
όσες σκηνές θέλετε, και να εναλλάσσετε μεταξύ τους κατά τη διάρκεια της μετάδοσης. Για να
γίνεται η εναλλαγή πιο ομαλά, μπορείτε να χρησιμοποιήσετε την λειτουργία \textit{Transition},
έχοντας ενεργοποιήσει την επιλογή \textit{Studio Mode}, η οποία βρίσκεται στο κάτω δεξιά μέρος του παραθύρου. Κάνοντας
\textit{Transition}, η εικόνα θα αλλάζει αργά και ομαλά από τη μία σκηνή στην άλλη και το τελικό
αποτέλεσμα θα είναι πιο επαγγελματικό.
\subsubsection{Sources}
Οι πηγές είναι τα διάφορα στοιχεία που μπορείτε να προσθέσετε στις σκηνές σας. Μπορείτε να
προσθέσετε κάμερες, οθόνες, εικόνες, κείμενο, και ήχο. Κάθε πηγή μπορεί να ρυθμιστεί ανάλογα
με τις ανάγκες σας, όπως η θέση, το μέγεθος, και η διαφάνεια, και πολλές ακόμα ιδιότητες που θα τις
δούμε στην αναλυτική προσέγγιση που θα κάνουμε την επόμενη Κυριακή. Μπορείτε να προσθέσετε όσες πηγές θέλετε σε
μια σκηνή και μπορείτε ακόμα να έχετε την ίδια πηγή σε πολλές σκηνές ταυτόχρονα, κάνοντας Copy μια πηγή
και Paste ως reference σε μια άλλη σκηνή (\texttt{Control + V}).

\subsubsection{Audio Mixer}
Το Audio Mixer είναι ο χώρος όπου μπορείτε να ρυθμίσετε τον ήχο των πηγών σας. Μπορείτε να αλλάξετε
την ένταση του ήχου, να ενεργοποιήσετε ή να απενεργοποιήσετε το μικρόφωνο, και να ρυθμίσετε τα όρια
του ήχου, δηλαδή πότε πρέπει να γίνεται clipping ή όχι. Επίσης, μπορείτε να προσθέσετε φίλτρα στον ήχο
σας, όπως noise suppression, noise gate, και compressor, για να βελτιώσετε την ποιότητα του ήχου σας μέσω
του equalizer.
\subsubsection{Settings}
Στις ρυθμίσεις, μπορείτε να διαμορφώσετε τις παραμέτρους του OBS Studio. Μπορείτε να αλλάξετε την
ανάλυση του βίντεο, το bit rate, το format του αρχείου, τον ήχο, και πάρα πολλά άλλα. Πιο προχωρημένα
ζητήματα για την παραμετροποίηση της καταγραφής, της ροής μας και του obs θα δούμε στην επόμενη συνεδρία.

\subsection{Ζητούμενα της σημερινής εργασίας}
Στη σημερινή εργασία, θα πρέπει να δημιουργήσετε μια σκηνή με μια κάμερα και ένα μικρόφωνο. Θα πρέπει να
επιλέξετε μια εικόνα background για να μην παραμείνει μαύρο το παρασκήνιο και να δημιουργήσετε μια σκηνή
με σκοπό να κάνετε μια ζωντανή μετάδοση στο YouTube. Επίσης θα χρειαστεί να καταγράψετε ένα βίντεο με τις σκηνές
σας κάνοντας εναλλαγή μεταξύ των σκηνών σας. Αναλυτικότερα:
\subsubsection{Δημιουργία πρώτης σκηνής}
Έχοντας ανοιχτή την αρχική οθόνη του OBS Studio, θα πρέπει να πατήσετε το κουμπί \textit{+} στην ενότητα
\textit{Scenes} για να δημιουργήσετε μια νέα σκηνή. Ονομάστε την σκηνή όπως θέλετε\footnote{Τυπικά ονόματα θα ήταν Παρουσίαση, Ομιλητής ή Screen share, αναλόγως με τον σκοπό της σκηνής}, και πατήστε \textit{OK}.
\subsubsection{Προσθήκη πηγών στη σκηνή που φτιάξαμε}
Θα προσθέσουμε μια κάμερα και ένα μικρόφωνο στη σκηνή που δημιουργήσαμε. Επίσης θα χρειαστεί να κατεβάσετε μια
εικόνα απο το Web για να τη χρησιμοποιήσετε ως background. Προτείνεται αυτή η εικόνα να είναι
ανοιχτού χρώματος, π.χ. κυανό ή άσπρο. Πατήστε το κουμπί \textit{+} στην
ενότητα \textit{Sources} και επιλέξτε \textit{Video Capture Device}\footnote{Σε περίπτωση που δεν έχετε κάμερα ή δεν έχει ο υπολογιστής webcam, μπορείτε να βάλετε ένα παράθυρο} για την κάμερα και \textit{Audio Input Capture}
για το μικρόφωνο. Ακολουθήστε τις οδηγίες που θα σας δώσει το OBS για να επιλέξετε την κάμερα και το μικρόφωνο (ή το παράθυρο).
\begin{info}
\textbf{Σημείωση:} Οι πηγές σε κάθε σκηνή μας στο OBS είναι σαν να βρίσκονται σε μια στοίβα.
Αυτό σημαίνει πως η πηγή που βρίσκεται πάνω στη στοίβα είναι αυτή που θα εμφανίζεται πιο πάνω απο τις άλλες πηγές.
Για να το διαπιστώσετε δοκιμάστε τι θα γίνει αν τραβήξετε την πηγή του background στο πάνω μέρος της στοίβας πηγών
και τι γίνεται αν την βάλετε στο κάτω μέρος.
\end{info}
\begin{questionbox}
\textbf{Ερώτηση:} Εμείς που θα θέλαμε να βάλουμε το background μας στην στοίβα πηγών;
\end{questionbox}
\subsubsection{Δημιουργία μιας δεύτερης σκηνής}
Ακολουθήστε τα παραπάνω βήματα για να δημιουργήσετε μια δεύτερη σκηνή, ονομάζοντας
την διαφορετικά από την πρώτη\footnote{Θα μπορούσε να είναι π.χ. Μόνο Ομιλητής ή Προεδρείο}.
\subsubsection{Προσθήκη πηγών στη δεύτερη σκηνή}
Σε αυτή τη σκηνή θα προσθέσουμε πάλι το ίδιο background που βάλατε πριν καθώς και ένα παράθυρο της επιλογής σας
απο τον υπολογιστή που χρησιμοποιείτε. Πατήστε το κουμπί \textit{+} στην ενότητα \textit{Sources} και επιλέξτε \textit{Window Capture}.
Ακολουθήστε τις οδηγίες που θα σας δώσει το OBS για να επιλέξετε το παράθυρο που θέλετε να προσθέσετε.
\subsubsection{Εναλλαγή μεταξύ των σκηνών}
\begin{info}
  \textbf{Σημείωση:} Για να γίνει μετάβαση σκηνών, πρέπει αρχικά να έχουμε ενεργοποιήσει τη λειτουργία
  \textit{Studio Mode},  που βρίσκεται στο κάτω δεξιά μέρος της αρχικής οθόνης του OBS.
\end{info}
Μετά την προσθήκη των πηγών στις σκηνές, θα πρέπει να τις εναλλάξετε μεταξύ τους. Αυτό μπορεί να γίνει με το πάτημα
του κουμπιού \textit{Transition} που βρίσκεται ενδιάμεσα των παραθύρων \textit{Preview} και \textit{Program}.
Μπορείτε να επιλέξετε τον τύπο μετάβασης που θέλετε, όπως \textit{Fade}, \textit{Cut}, και \textit{Slide},
και να ρυθμίσετε τη διάρκεια της μετάβασης σε ms. Τέλος βάλτε και έναν τίτλο κάτω απο την κάμερα σας,
ώστε να είναι πιο εύκολο να αναγνωρίζεται ο ομιλητής. Αυτό μπορείτε να το κάνετε προσθέτοντας μια νέα πηγή
τύπου \textit{Text} και γράφοντας το όνομα του ομιλητή σας\footnote{John Doe ή Jane Doe}.
\subsubsection{Καταγραφή βίντεο}
Για να καταγράψετε το βίντεο με τις σκηνές σας, θα πρέπει να πατήσετε το κουμπί \textit{Start Recording} που βρίσκεται
στο κάτω δεξιά μέρος του παραθύρου. Το βίντεο θα αποθηκευτεί στον υπολογιστή σας στον φάκελο που έχετε ορίσει στις ρυθμίσεις.
(Περισσότερα για της ρυθμίσεις θα δούμε στο επόμενο session)
\subsubsection{Ζωντανή μετάδοση στο YouTube}
Θα πατήσουμε το κουμπί \textit{Start Streaming} που βρίσκεται στο κάτω δεξιά μέρος του παραθύρου. Θα μας ζητηθεί να
επιλέξουμε την πλατφόρμα μετάδοσης, που στην περίπτωση μας είναι το YouTube, και θα μας ζητηθεί να εισάγουμε το κλειδί
το οποίο θα σας δοθεί εκείνη τη στιγμή. Αφού εισάγετε το κλειδί, θα πατήσετε \textit{OK} και η μετάδοση θα είναι έτοιμη να
ξεκινήσει. Εσείς αρκεί απλώς να πατήσετε ξανά το κουμπί \textit{Start Streaming} όταν είστε έτοιμοι.
\begin{info}
\textbf{Προσοχή:} Είναι σημαντικό να έχουμε μια σταθερή σύνδεση στο Internet για να μην υπάρξουν προβλήματα κατά τη μετάδοση.
Ένας καλός ρυθμός upload που θα θέλαμε να έχουμε για μια ροή 1080p είναι τουλάχιστον 5-10 Mbps.
Μπορούμε να ελέγξουμε την ταχύτητα μεταφόρτωσης μας στο \url{https://www.speedtest.net/}.
\end{info}






\section*{Λεπτομέρειες Υποβολής (σε περίπτωση που δεν προλάβετε να τελειώσετε ή δεν είστε στο LAB)}
\noindent Αφήστε με να περάσω απο το θρανίο σας και να ελέγξω ότι οι σκηνές σας είναι σωστές.
Σε περίπτωση που δεν προλάβετε να τελειώσετε την εργασία σας, μπορείτε να μου στείλετε
screenshots της κεντρικής οθόνης του OBS και των ρυθμίσεων που έχετε κάνει. Επίσης θα χρειαστεί να
Κάνετε εξαγωγή τις σκηνές σας, πατώντας το Tab \textbf{Scenes\footnote{ή Σκηνές}} και έπειτα
\textbf{Export Scene}. Στη συνέχεια, θα πρέπει να μου στείλετε το \texttt{.JSON} αρχείο που θα δημιουργηθεί,
το οποίο θα είναι στην ουσία η παραμετροποίηση των σκηνών σας.
\begin{info}
  \textbf{Προσοχή:} Το αρχείο που θα εξαχθεί δεν περιέχει τις πηγές σας αυτούσιες, αλλά μόνο τις ρυθμίσεις που έχετε κάνει.
  Αυτό σημαίνει πως θα πρέπει να μου στείλετε και τις \textbf{Πηγές σας\footnote{ή Sources}}, όπως τις έχετε δημιουργήσει, σε μορφή screenshot.
\end{info}
\section*{Επικοινωνία}
Για οποιαδήποτε απορία ή διευκρίνιση, μπορείτε να επικοινωνήσετε μαζί μου στο \href{mailto:dit17046@uop.gr} ή μέσω της
ομάδας μας στο Microsoft Teams.
\end{document}
