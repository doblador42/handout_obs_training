\documentclass[12pt,a4paper]{article}
\usepackage{fullpage}
% Language and font settings
\usepackage{polyglossia}
\setdefaultlanguage{greek}      % Main language
\setotherlanguage{english}      % Secondary language
\usepackage{fontspec} % Required for font selection in lualatex
\usepackage{csquotes}
\usepackage{libertine}
% Formatting and lists
\usepackage{enumitem} % Customizable lists
\setlist{noitemsep}   % Remove extra space between list items
% Boxes and highlights
\usepackage{tcolorbox}
\usepackage{hyperref}
\hypersetup{
  colorlinks=true,
  linkcolor=blue,
  filecolor=magenta,      
  urlcolor=cyan,
  citecolor=blue,
  pdfpagemode=FullScreen,
}
\tcbuselibrary{listings, breakable}

\newtcolorbox{info}{colback=blue!5!white, colframe=blue!75!black}
\newtcolorbox{questionbox}{colback=green!5!white, colframe=green!75!black, breakable}
\usepackage{amsmath, amssymb} % Math support
\title{\textbf{εκπαιδευτικό φυλλάδιο για το OBS}}
\author{Ερμής Δούλος (\href{mailto:dit17046@uop.gr}{dit17046@uop.gr})\thanks{Επιβλέπων καθηγητής: Ιωάννης Μοσχολιός (\href{mailto:idm@uop.gr}{idm@uop.gr})}}
\date{\today}

\begin{document}
\maketitle
\begin{center}
\includegraphics[width=0.5\textwidth]{Figures/uop.png} 
\end{center}
\noindent\makebox[\linewidth]{\rule{\textwidth}{1.5pt}}

\section*{Εισαγωγή}
Σύντομη περιγραφή του σκοπού της εργασίας

\section{Οδηγίες}
\begin{itemize}
    \item Διαβάστε προσεκτικά κάθε ερώτηση.
    \item Συμπληρώστε τις απαντήσεις σας καθαρά και με σαφήνεια.
\end{itemize}

\section{Ερωτήσεις}
\begin{enumerate}[label=\textbf{Ερώτηση \arabic*.}]
    \item Περιγράψτε τον όρο «αλγόριθμος».
    \item Αναλύστε τη σημασία της ανάλυσης πολυπλοκότητας.
\end{enumerate}
\begin{info}
  Καταληκτική ημερομηνία: 15 Δεκεμβρίου 2024
\end{info}
\begin{questionbox}
\textbf{Ερώτηση 1:} Περιγράψτε τη διαφορά μεταξύ στοίβας (stack) και ουράς (queue).
\end{questionbox}
\section*{Λεπτομέρειες Υποβολής}
Αφήστε με να περάσω απο το θρανίο σας και να ελέγξω ότι οι σκηνές σας είναι σωστές.
Σε περίπτωση που δεν προλάβετε να τελειώσετε την εργασία σας, μπορείτε να μου στείλετε
screenshots της κεντρικής οθόνης του OBS και των ρυθμίσεων που έχετε κάνει. Επίσης θα χρειαστεί να 
κάνετε εξαγωγή τις σκηνές σας, πατώντας το Tab \textbf{Scenes\footnote{ή Σκηνές}} και έπειτα
\textbf{Export Scene}. Στη συνέχεια, θα πρέπει να μου στείλετε το \texttt{.json} αρχείο που θα δημιουργηθεί,
το οποίο θα είναι στην ουσία η παραμετροποίηση των σκηνών σας.
\begin{info}
\textbf{Προσοχή:} Το αρχείο που θα γίνει εξαχθεί δεν περιέχει τις πηγές σας αυτούσιες, αλλά τις ρυθμίσεις που έχετε κάνει.
Αυτό σημαίνει πως θα πρέπει να μου στείλετε και τις πηγές σας, όπως τις έχετε δημιουργήσει, σε screenshot.
\end{info}
\end{document}
